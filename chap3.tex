%chap3.tex
%机器人运动学
\chapter{机器人运动学}
\section{机器人运动方程的表示}
\subsection{机械手的运动姿态和方向表示}
\subsubsection{机械手的运动方向}
\begin{equation}
\T_6=
\begin{bmatrix}
n_x & o_x & a_x & p_x\\
n_y & o_y & a_y & p_yz\\
n_z & o_z & a_z & p_z\\
0 & 0 & 0 & 1
\end{bmatrix}
\end{equation}
\subsubsection{用欧拉变换表示运动姿态}
\begin{equation}
Euler(\phi, \theta, \psi)=Rot(z, \phi)Rot(y, \theta)Rot(z, \psi)
\end{equation}
\subsubsection{用PRY组合变换表示运动姿态}
\begin{equation}
RPY(\phi, \theta, \psi)=Rot(z, \phi)Rot(y, \theta)Rot(z, \psi)
\end{equation}
\begin{center}
\begin{description}
\item[R] rool(横流)
\item[P] pitch(俯仰)
\item[Y] yaw(偏转)
\end{description}
\end{center}
\section{平移变换的不同坐标表示}
一旦机械手的运动姿态由某个姿态变换规定之后,它在基系中的位置就能够由左乘一个对英语矢量$\bm{p}$的变换来确定。
\begin{equation}
T_0=
\begin{bmatrix}
1 & 0 & 0 & p_x\\
0 & 1 & 0 & p_y\\
0 & 0 & 1 & p_z\\
0 & 0 & 0 & 1
\end{bmatrix}
[\text{某姿态变换}]
\end{equation}
\subsubsection{用柱面坐标表示运动位置}
\begin{equation}
Cyl(z, \alpha, \gamma)=Trans(0, 0, z)Rot(z, \alpha)Trans(r, 0, 0)
\end{equation}
\subsubsection{用球面坐标表示运动位置}
\begin{equation}
Sph(\alpha, \beta, \gamma)=Rot(z, \alpha)Rot(y, \beta)Trans(0, 0, \gamma)
\end{equation}