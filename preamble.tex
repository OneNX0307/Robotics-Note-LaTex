%preamble.tex
\usepackage[left=1.5in,right=1.5in,top=1in,bottom=1in]{geometry} %页边距
\renewcommand{\baselinestretch}{1.25}
\usepackage{indentfirst}
	\setlength{\parindent}{2em}
\usepackage{amsmath}
\usepackage{bm}
\setlength{\parskip}{1em}%段间距
\renewcommand{\contentsname}{目 \quad 录}
\renewcommand{\today}{\number\year 年\number\month 月\number\day 日}
\usepackage{graphicx}%插图
\usepackage{xeCJK}
\usepackage{fontspec}
%----------------字体-------------------------
\setmainfont{Times New Roman}
\setsansfont{Arial}
\setCJKmainfont[BoldFont=simhei.ttf,ItalicFont=simkai.ttf]{simsun.ttf}


%---------页眉页脚--------------------------
%\usepackage{fancyhdr}%页眉页脚
%    \pagestyle{fancy}
%    \fancyhf{}
%    \fancyhead[ER]{leftmark}%偶数页右页眉
%    \fancyhead[OL]{rightmark}%奇数页左页眉
%    \fancyhead[EL,OR]{$\cdot$\ \thepage\ $\cdot$}%偶数页的左页脚,奇数页的右页脚
%    \renewcommand{\headrulewidth}{0.4pt}%页眉线宽度

%------------------------------------------
\usepackage{fancyhdr}
\pagestyle{fancy}
\renewcommand{\chaptermark}[1]{\markboth{#1}{}}
\renewcommand{\sectionmark}[1]{\markright{\thesection\ #1}}
\fancyhf{} % 清空当前的页眉页脚
\fancyfoot[C]{\bfseries\thepage}
\fancyhead[LO]{\bfseries\rightmark}
\fancyhead[RE]{\bfseries\leftmark}
\renewcommand{\headrulewidth}{0.4pt}
\renewcommand{\footrulewidth}{0pt}
%-------------章节格式--------------------------
\usepackage{titlesec}%设置章节格式
    \titleformat{\chapter}{\centering\Huge\bfseries}{第\,\thechapter\,章}{1em}{}
    \renewcommand{\sectionmark}[1]{\markright{\small\thesection\quad #1}{}}
    %以上两个重定义语句一定要放在\pagestyle{fancy}之后,因为在 fancyhdr 宏包中对这两个命令重新进行了定义,在 \pagestyle{fancy} 之后重定义它们就可将 fancyhdr 中的相应命令覆盖掉。

\usepackage{titletoc}%调整章节标题在目录页中的格式
\newcommand{\upcite}[1]{\textsuperscript{\cite{#1}}}%参考文献在右上角
\newcommand{\zsb}[2]{{}^{#2}{\bm{#1}}}
\newcommand{\zb}[3]{{}^{#2}_{#3}{\bm{#1}}}
\newcommand{\djb}[3]{{}^{#2}{\bm{#1}}_{#3}}
\newcommand{\rotx}{\begin{equation}
\bm{R}(x,\theta)=
\begin{bmatrix}
1 & 0 & 0\\
0 & c\theta & -s\theta\\
0 & s\theta & c\theta
\end{bmatrix}
\end{equation}}

\newcommand{\roty}{\begin{equation}
\bm{R}(y,\theta)=
\begin{bmatrix}
c\theta & 0 & s\theta\\
0 & 1 & 0\\
-s\theta & 0 & c\theta
\end{bmatrix}
\end{equation}}
\newcommand{\rotz}{\begin{equation}
\bm{R}(z,\theta)=
\begin{bmatrix}
c\theta & -s\theta & 0\\
s\theta & c\theta & 0\\
0 & 0 & 1
\end{bmatrix}
\end{equation}}
\newcommand{\Rotx}{\begin{equation}
\bm{Rot}(x,\theta)=
\begin{bmatrix}
1 & 0 & 0 & 0\\
0 & c\theta & -s\theta & 0\\
0 & s\theta & c\theta & 0\\
0 & 0 & 0 & 1
\end{bmatrix}
\end{equation}}

\newcommand{\Roty}{\begin{equation}
\bm{Rot}(y,\theta)=
\begin{bmatrix}
c\theta & 0 & s\theta & 0\\
0 & 1 & 0 & 0\\
-s\theta & 0 & c\theta 0\\
0 & 0 & 0 & 1\\
\end{bmatrix}
\end{equation}}
 
\newcommand{\Rotz}{\begin{equation}
\bm{Rot}(z,\theta)=
\begin{bmatrix}
c\theta & -s\theta & 0 & 0\\
s\theta & c\theta & 0 & 0\\
0 & 0 & 1 & 0\\
0 & 0 & 0 & 1
\end{bmatrix}
\end{equation}}
